The sticker system is a formal DNA computing model introduced by 
\cite{kari1998,paun1998}.
The sticker operation used in the sticker system  is the most 
basic and essential operation based on using Watson-Click 
complementarity and sticky ends.

In this paper, we introduce Haskell program modules for realize 
and analyse features related to the sticker system.
Most of realized operations and constructions are based on
the facts in \cite{paun1998}, but we modify the expression of
domino and sticker operations for realizing Haskell functions.

We modify the definition of domino ($D$) from a string of pairs
of alphabet to a triple $(l,r,x)$ of two string $l$, $r$
and an integer $x$.
For example 
$\left(\begin{array}{c} \lambda \\ C \end{array}\right)$
$\left[\begin{array}{c} AT \\ TA \end{array}\right]$
$\left(\begin{array}{c} GC \\ \lambda \end{array}\right)$
is represented by $(ATGC,CTA,-1)$.
According to this modification, the sticker operation 
$\mu:D \times D \to D \cup \{\bot\}$ is reformed to
one equation 
$\mu((l_1,r_1,x_1),(l_2,r_2,x_2)) =(l_1l_2,r_1r_2,x_1)$,
if $(l_1l_2,r_1r_2,x)\in D$ and 
$x_1+ length(r_1)-length(l_1) = x_2$.
Using this simple representation of domino and sticker operations,
we implement them in Haskell.

One of the benefits of using Haskell language 
is it has descriptions for infinite set of strings 
using lazy evaluation schemes.
For example, the infinite set $\{a,b\}^*$ is 
denoted by finite length of expression \verb|(sstar ['a','b'])|.
Using \verb|(take)| function to view contents of an infinite set
(e.g. \verb|(take 5 (sstar ['a','b']))| is \verb|["","a","b","aa","ba"]|).
Further using set theoretical notions in Haskell, we can easily
realize the definitions of various kinds of set of domino.
For example, to make a sticker system which generate the equivalent 
language of a finite automaton we need an atom set
$$
A_2 = \bigcup_{i=1}^{k+1} 
\{(xu,x,0)|x\in \Sigma^{\le (k+2-i)}, u\in \Sigma^i,
\delta^*(0,xu)=i-1 \}.
$$
In Haskell notations, we have following function definitions.
{\small
\begin{quote}
\begin{verbatim}
aA2::Automaton->[Domino]
aA2 m@(q,s,d,q0,f) = concat [(aA2' m i)| i<- [1..(k+1)]] where k = (length q)-1
aA2'::Automaton->Int->[Domino]
aA2' m@(q,s,d,q0,f) i =  [(x++u,x,0)| (x,u) <- xupair, (dstar d 0 (x++u)) ==(i-1)]
       where xupair = [(x,u)|x<-(sigmann s (k+2-i)), u<-(sigman s i)]
             k = (length q)-1
\end{verbatim}
\end{quote}
}

The modules is composed of 4 parts,
{\bf Automaton} module
(string, automaton and their languages),
{\bf Sticker} module
(domino, sticker system and their languages),
{\bf Grammar} module
(context-free grammar, linear-grammar and their languages),
and {\bf \LaTeX} output module
(pretty printing for domino).
Using our module functions, we can easily define finite automata
and linear grammars and construct sticker systems which have the
same power of finite automata and linear grammars
introduced in \cite{paun1998}.

In this paper, we introduce implemented module functions and
examples of sticker systems with equivalent power of 
concrete finite automata and linear grammars.
